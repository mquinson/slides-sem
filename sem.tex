% handout
\documentclass[10pt,final,usepdftitle=false]{beamer}
\mode<presentation>
\usetheme{default}
\usepackage[french]{babel}
\usepackage[utf8]{inputenc}

\hypersetup{pdftitle={Sciences en Marche}} \title[Sciences en Marche]{Sciences
  en Marche\\[0.5em]Les \textit{raisons} de la colère}
\author{M. Quinson\\[0.5em]
  \small  (merci à \textbf{L. Nussbaum} pour la plupart de ces transparents)\\[0.5em]
  \url{https://github.com/mquinson/slides-sem}} \date{} \usepackage{eurosym}

\setbeamersize{text margin left=1em,text margin right=1em}

\begin{document}

\frame{\titlepage}

\begin{frame}{Notre métier: enseignant-chercheur}
\begin{itemize}
\item 50\% enseignant, 50\% chercheur
\end{itemize}
\end{frame}

\begin{frame}{Notre métier: \underline{enseignant}-chercheur}
\begin{itemize}
\item Donner des cours, 192 heures (de TD) par an
\item Préparer et mettre à jour les supports (veille nécessaire)
\item Préparer les examens, évaluer les étudiants
\item Participer aux jurys
\item Gérer les problèmes divers des étudiants
\item Coordonner les interventions des intervenants professionnels
\item Participer à l'évolution des formations (réunions pédagogiques, dossiers
  d'habilitation, discussions sur le contenu des modules, etc.)
\item Participer à l'administration des composantes d'enseignement
\end{itemize}
\end{frame}

\begin{frame}{Notre métier: enseignant-\underline{chercheur}}
\begin{itemize}
\item Faire de la recherche: augmenter la masse de savoir humain
  \begin{itemize}
  \item Soi-même et avec les collègues (réunions)
  \item En encadrant des doctorants, ingénieurs, stagiaires
  \end{itemize}
  \smallskip
\item Diffuser: Écrire des publications scientifiques, les présenter en
  conférence
\item Financer: Écrire des propositions de projets nationaux (ANR) ou européens
  \begin{itemize}
  \item Très chronophage: des semaines par dossier; taux d'acceptation $<$ 10\%
  \item Notre salaire est assuré, mais le financement de base des labos ne
    suffit pas
  \end{itemize}
\item Évaluer: Participer à l'évaluation des projets et travaux des autres
  \begin{itemize}
  \item Relectures et expertises d'articles, thèses et HDR
  \item Faire du reporting sur les projets acceptés, Expertiser le reporting des
    autres
  \end{itemize}
\item Administrer l'université
  \begin{itemize}
  \item Il faut gérer et répartir les moyens obtenus, etc
  \item Nombreuses structures collégiales: {\small équipe de recherche,
      département, labo \ldots}
  \item Également les grandes infrastructures de recherche 
  \end{itemize}

\end{itemize}
\end{frame}

\begin{frame}{Notre métier: enseignant-chercheur}
\begin{itemize}
\item Dans l'ensemble, un métier:
\begin{itemize}
\item Intéressant, varié, à la pointe de la connaissance et de la technologie
\item Extrêmement prenant (les semaines font souvent 50 heures ou plus)
%\item Où on a l'impression d'être utile
\item Pas trop mal payé (2300\euro{} net par mois après 10 ans et un doctorat)
\item Avec beaucoup de liberté
\end{itemize}
\end{itemize}

\pause \bigskip %

\centerline{\large Mais le monde de l'ESR\footnote{ESR: Enseignement Supérieur
    et Recherche} change \ldots}

\end{frame}

\begin{frame}{Difficultés budgétaires des universités} 
\begin{itemize}
\item 2007: Loi relative aux libertés et responsabilités des universités (LRU)
\item Transfert de \textsl{Responsabilités et Compétences Élargies} (RCE) aux
  universités
  \begin{itemize}
  \item Notamment la gestion de la masse salariale (plus fonctionnaires d'état)
  \item Gros bouleversements et pas mal de problèmes
  \end{itemize}

\item Notre pire problème: \alert{Glissement Vieillesse Technicité (GVT)}
  \begin{itemize}
  \item Augmentation de la masse salariale avec l'ancienneté des personnels
  \item Peu sensible quand la pyramide des âges est équilibrée (EDF);\\
    Important à l'université qui a grossi par grosses vagues irrégulières
    ($\rightarrow$2019)
  \item RCE d'état aux universités à \alert{budget constant},
    sans compter le GVT
  \end{itemize}
\pause
\item Les universités étranglées par la masse salariale suppriment des postes
  \begin{itemize}
  \item Près de 50\% des postes à pourvoir l'année dernière à l'UL
  \item UL'2015: GVT = 4,8M\euro{}; compensation espérée = 0,5M\euro{}.
  \item UL'2015: 5M\euro{} manquent au total, et il faut \textit{geler} plus de
    la moitié de la MS
  \end{itemize}
\end{itemize}
\end{frame}


\begin{frame}{Difficultés budgétaires de l'ESR}
\begin{itemize}
\item Mais la ministre dit créer 1000 postes!
  \begin{itemize}
  \item Elle \textit{autorise} les universités à créer des postes de
    fonctionnaires (+60M\euro)
  \item Cela ne suffit pas à combler le GVT dans les universités
  \item Projet de loi finance 2015: nouvelle baisse annoncée de 70M\euro{}
  \end{itemize}
\pause
\item Au CNRS: tous les départs à la retraite sont renouvelés (et seulement eux)
  \begin{itemize}
  \item Budget ANR en chute; pression accrue pour récupérer le budget à l'europe
  \end{itemize}
\end{itemize}
\end{frame}

\begin{frame}{Difficultés budgétaires: conséquences}
\begin{itemize}
\item Il faut répartir la charge de travail des non-remplacés sur leurs collègues
\item Heures supplémentaires (\textsl{complémentaires}) subies
  \begin{itemize}
  \item Menace de l'équilibre ``50\% enseignement / 50\% recherche''
  \item UL'2015: budget ``Heures complémentaires'' à réduire drastiquement
  \end{itemize}
\item Intervenants extérieurs supplémentaires à trouver%\\
%	($\approx$25\% des heures au département Informatique)
\item De moins en moins de postes permanents:
\begin{itemize}
\item De plus en plus de CDD (précarité)
\item Pour des besoins permanents (ex: fonction support): re-former sans arrêt
\end{itemize}
\medskip
\item Moins de recherche en France ne présage rien de bon pour le XXIième
  \begin{itemize}
  \item En pratique, on cherche plus souvent de l'argent qu'autre chose
  \end{itemize}
\end{itemize}
\end{frame}

\begin{frame}{Mauvaises solutions}
\begin{itemize}
\item Diminuer la qualité de l'enseignement (\structure{solution classique})\\
  {\small (groupes plus nombreux, un enseignant pour deux groupes de TP, etc.)}
\item Réduire les effectifs (supprimer des groupes d'étudiants)
\item Fermer des formations (même si on peut encore rationnaliser un peu)
\item Sélectionner à l'entrée de la licence
\item Ajouter et resserrer des \textsl{numerus clausus}
\item Augmenter les frais d'inscription à l'université %
  ($\times$2 à $\times$4 à Mines-Telecom)
%\item Contribution exceptionnelle des étudiants
\item Modulation de service: les E/C ne font plus 192h mais plus (ou moins)
\end{itemize}
\end{frame}

\begin{frame}{Vraies solutions ?}
\begin{itemize}

\item Rediriger une partie du financement du Crédit Impôt Recherche ?
\begin{itemize}
\item Crédit d'impôt pour financer la R\&D des entreprises (ce qui est bien)
\item Inefficace d'après la Cour des Comptes:\\
  {\sl \og L’évolution qu’a connue la dépense intérieure de R\&D des entreprises
    n’est pas à ce jour en proportion de l’avantage fiscal accordé aux
    entreprises. \fg }
\item Trop peu contrôlé, nombreuses «optimisations fiscales» démontrées
\item Environ 6$\,$000 M\euro/an ($\approx$ Cnrs+Inserm+CEA+Inra+Inria+IRD+
  Ifremer, etc)
\item Manque au budget de l'ESR: 3\% de la somme
\item Fin de non-recevoir car \textit{«dispositif central de l'attactivité du
    pays»}
\end{itemize}
\pause \bigskip
\item Considérer qu'un ESR de qualité est important (surtout en période de
  crise)
  \begin{itemize}
  \item Peu importe d'où vient l'argent au fond.
  \item Portiques ecotaxe: 2$\,$000 M\euro{} perdus; Besoins vitaux ESR:
    250M\euro
  \end{itemize}
\end{itemize}
%\bigskip
\begin{center}
  Sous-financement des hôpitaux = danger immédiat\\
  Sous-financement de l'ESR = futur hypothéqué
\end{center}
\end{frame}

\end{document}

%  LocalWords:  ESR l'ESR
